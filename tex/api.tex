\documentclass{article}
\usepackage[letterpaper, margin=1.0in]{geometry}

\begin{document}
\noindent
Robert Scanlon \\
CS 496 Fall 2017 \\
Final Project

\section*{People Objects}
People objects are returned with the following properties. \\
\begin{tabular}{| l | l | l |}
	\hline
	\textbf{Property}     & \textbf{Type}   & \textbf{Description} \\
	\hline
	id                    & String          & url safe identifier \\
	\hline
	self                  & Url String      & link to self (not an intrisic property of People objects; \\
	                      &                 & added when request is made) \\
	\hline
	fname		      & String          & first name of person \\
	\hline
	lname	              & String          & last name of person \\
	\hline
	age	              & Integer         & age of person (years) \\
	\hline
	address               & String          & home address of person \\
	\hline
	pets	              & String[]        & a list containing the ids of pets for which this \\
	                      &                 & person is caretaker for \\
	\hline
\end{tabular}

\section*{Pet Objects}
Pet objects are returned with the following properties. \\
\begin{tabular}{| l | l | l |}
	\hline
	\textbf{Property}     & \textbf{Type}   & \textbf{Description} \\
	\hline
	id                    & String          & url safe identifier \\
	\hline
	self                  & Url String      & link to self (not an intrisic property of Pet objects; \\
	                      &                 & added when request is made) \\
	\hline
	name		      & String          & name of pet \\
	\hline
	species               & String          & species of pet (cat, dog, etc.) \\
	\hline
	age                   & Integer	        & age of pet (years) \\
	\hline
	weight                & Integer         & weight of pet \\
	\hline
	caretaker             & String          & id of the person who is caretaker for this pet, \\
	                      &                 & NULL if the pet does not curretly have a caretaker \\
	\hline
\end{tabular}

\section*{Get all People and Pets}
\texttt{GET /} \\
\textbf{Note:} Returns a json object containing "People" (a json list of all People in the
database) and "Pets" (a json list of all Pets in the database).

\section*{Get all People}
\texttt{GET /person} \\
\textbf{Note:} Returns a json list of all People in the database.

\section*{Get a single Person}
\texttt{GET /person/:id} \\
\textbf{Note:} A request containing a valid id string which does not
identidy a Person in the database will return a 404 status code. A request
containing an invaid id string will return a status code 500 server error.

\section*{Get all Pets}
\texttt{GET /pet} \\
\textbf{Note:} Returns a json list of all Pets in the database.

\section*{Get single Pet}
\texttt{GET /pet/:id} \\
\textbf{Note:} A request containing a valid id string which does not
identidy a Pet in the database will return a 404 status code. A request
containing an invaid id string will return a status code 500 server error.

\section*{Create a new Person}
\texttt{POST /person} \\
\textbf{Input} : application/json \\
\begin{tabular}{| l | l | l |}
	\hline
	\textbf{Name} & \textbf{Type} & \textbf{Description} \\
	\hline
	fname         & string        & \textbf{Required.} First name of Person \\
	\hline
	lname         & string        & \textbf{Required.} Last name of Person \\
	\hline
	age           & integer       & \textbf{Required.} Age of Person (years) \\
	\hline
	address       & integer       & \textbf{Required.} Address of Person \\
	\hline
\end{tabular} \\
\textbf{Note:} A request which contains invalid body data will return a
status code 400 Bad Request.

\section*{Create a new Pet}
\texttt{POST /pet} \\
\textbf{Input} : application/json \\
\begin{tabular}{| l | l | l |}
	\hline
	\textbf{Name} & \textbf{Type} & \textbf{Description} \\
	\hline
	name          & String        & \textbf{Required.} Name of pet \\
	\hline
	species       & String        & \textbf{Required.} Species of Pet \\
	\hline
	age           & integer       & \textbf{Required.} Age of Pet (years) \\
	\hline
	weight        & integer       & \textbf{Required.} Weight of Pet (lbs) \\
	\hline
\end{tabular} \\
\textbf{Note:} A request which contains invalid body data will return a
status code 400 Bad Request.

\section*{Modify a Person}
\texttt{PATCH /person/:id} \\
\textbf{Input} : application/json \\
\begin{tabular}{| l | l | l |}
	\hline
	\textbf{Name} & \textbf{Type} & \textbf{Description} \\
	\hline
	age           & integer       & New age of Person (years) \\
	\hline
	address       & string        & New address of Person \\
	\hline
\end{tabular}
\\
Note: The \texttt{id}, \texttt{fname}, \texttt{lname}, and \texttt{pets}
Person attributes cannot be modified through this request.

\section*{Modify a Pet}
\texttt{PATCH /pet/:id} \\
\textbf{Input} : application/json \\
\begin{tabular}{| l | l | l |}
	\hline
	\textbf{Name} & \textbf{Type} & \textbf{Description} \\
	\hline
	age           & integer        & New age of Pet (years) \\
	\hline
	weight        & integer        & New weight of Pet (lbs) \\
	\hline
\end{tabular}
\\
Note: The \texttt{id}, \texttt{name}, \texttt{species} and \texttt{caretaker} 
Pet attributes cannot be modified though this request.

\section*{Delete a Person}
\texttt{DELETE /person/:id} \\
Note: The caretaker attribute of all Pets for which this Person was caretaker for
is set to \texttt{NULL} upon the deletion of this Person.

\section*{Delete a Pet}
\texttt{DELETE /pet/:id} \\
Note: This Pet's id string is removed from a Person's pets[] attribute upon
deletion of this Pet, if said person was caretaker for this Pet.

\section*{Get All Pets With No Caretaker}
\texttt{GET /pet/free} \\
Note: This request will return a JSON array of all Pets in the database whose
\texttt{caretaker} attribute is \texttt{NULL}.

\section*{Add Care Relationship}
\texttt{PUT /pet/:id/caretaker} \\
\textbf{Input} : application/json \\
\begin{tabular}{| l | l | l |}
	\hline
	\textbf{Name} & \textbf{Type} & \textbf{Description} \\
	\hline
	person\_id   & string         & \textbf{Required.} Person id string of the Person \\
		     &                & who will be caretaker for this Pet. \\
	\hline
\end{tabular} \\
Note: If the Pet identified by the \texttt{id} in the url already has a caretaker,
the \texttt{id} in the url does not match a Pet in the databse, or the \texttt{id}
passed in the request body does not match a person in the databse, a \texttt{400} error code
is returned along with an appropriate error message. If the request is successful the Pet's
\texttt{caretaker} is updated, along with the Person's \texttt{pets} list.

\section*{Remove Care Relationship}
\texttt{PATCH /pet/:id/caretaker} \\
Note: If the \texttt{id} passed in the url does not identify a Pet in the databse,
or the person \texttt{id} of this Pets \texttt{caretaker} does not identify a Person
in the databse, a \texttt{400} error is returned with the appropriate message.
If the request is successful, the Pet's \texttt{caretaker} attribute is set to
\texttt{NULL} and the Pet's \texttt{id} string is removed from the Person's
\texttt{pets} list.

\end{document}
